%%%%%%%%%%%%%%%%%%%%%%%%%%%%%%%%%%%%%%%%%
% Original author:
% Linux and Unix Users Group at Virginia Tech Wiki 
% (https://vtluug.org/wiki/Example_LaTeX_chem_lab_report)
%
% License:
% CC BY-NC-SA 3.0 (http://creativecommons.org/licenses/by-nc-sa/3.0/)
%
%%%%%%%%%%%%%%%%%%%%%%%%%%%%%%%%%%%%%%%%%

%----------------------------------------------------------------------------------------
%	PACKAGES AND DOCUMENT CONFIGURATIONS
%----------------------------------------------------------------------------------------

\documentclass[a4paper,11pt]{article}

\usepackage{graphicx} % Required for the inclusion of images
\usepackage{titling}
\usepackage{upgreek}

\setlength{\droptitle}{-13em}   % This is your set screw

\setlength\parindent{2em} % Removes all indentation from paragraphs

\renewcommand{\labelenumi}{\alph{enumi}.} % Make numbering in the enumerate environment by letter rather than number (e.g. section 6)

\usepackage{times} % Uncomment to use the Times New Roman font

\usepackage{indentfirst}

\setlength{\parskip}{0.5em plus1pt minus1pt}

%----------------------------------------------------------------------------------------
%	DOCUMENT INFORMATION
%----------------------------------------------------------------------------------------

\title{SDP Group 8: Final Individual Report} % Title

\author{Blake Hawkins} % Author name

\date{21 April, 2014} % Date of the milestone

\begin{document}

\maketitle % Insert the title, author and date

\begin{center}
Mentor: Katharina Heil, Guest Mentor: Tom Spink % Instructor/supervisor
\\
Members: Blake Hawkins, % Partner names
Lubomir Vikev,
Borislav Ikonomov,
Yordan Stoyanov,
James Linehan,
Lukas Dirzys,
Iain Brown,
Emanuel Martinov,
Robaidh Mackinnon,
Aneesh Ghosh

\end{center}

%----------------------------------------------------------------------------------------
%   SECTION 1 (Introduction)
%----------------------------------------------------------------------------------------

\section{Introduction}

My contributions to our project spanned many topics, notably software design, physics simulation, project management, and strategy. This report is loosely divided into sections for each of these topics, in which I describe some contributions and offer commentary.

%----------------------------------------------------------------------------------------
%	SECTION 2 (Software Design Contributions)
%----------------------------------------------------------------------------------------

\section{Software Design Contributions}

My recent works in software design include a major refactoring to our monolithic \textbf{Strategy} class, in which I split it into two more cohesive classes: \textbf{Strategy}, which runs our GUI components and vision/world state; and \textbf{StrategyThread}, which is an interruptible state machine that keeps both robots updated with the latest commands. This refactoring also opened our strategy system to unit tests by completely eliminating \textit{static} references. I also wrote a total of 19 unit tests related to ball prediction, our \textbf{minorHull()} algorithm, and various vector mathematics methods.

I also conducted many smaller refactorings, wrote the majority of our project's JavaDocs, made the initial transition from Milestone 3 to our \textbf{Strategy/Robot} class pair, thereby successfully providing abstract commands available to both attacker and defender. Further, I fixed many regressions left over by large refactorings written by others.

In relation to code re-usability, I believe these contributions are the most significant. 

%----------------------------------------------------------------------------------------
%	SECTION 3 (Physics Simulation Contributions)
%----------------------------------------------------------------------------------------

\section{Physics Simulation Contributions}

I wrote a large ball prediction system which supported the ball reflecting off multiple boundary wells, provided abstract values for coefficients of restitution and friction, and distinguished goal lines from normal walls.

During milestone 1 I also wrote an odometry module which allowed us to estimate our robot location after an arbitrary distance traveled, and any given number of turns. However, due to imprecision in our robot, the system was less accurate than a much simpler solution involving timers.

%----------------------------------------------------------------------------------------
%	SECTION 4 (Project Management Contributions)
%----------------------------------------------------------------------------------------

\section{Project Management Contributions}

Acted as a project manager in cases where having one was academically necessary, but our group mostly worked cohesively, with everyone administering their own tasks and availability. I provided some additional support by partially managing our issue tracker and writing tutorials on our team wiki.

I also used paired programming on some occasions as a way of staying on track with everyone's activities, and found the process surprisingly useful and efficient. In addition, I worked on other tasks like presentation design and speech planning.

%----------------------------------------------------------------------------------------
%	SECTION 5 (Strategy Contributions)
%----------------------------------------------------------------------------------------

\section{Strategy Contributions}

I spent a large fraction of my time developing various abstract methods for our robot pair to execute strategy, such as \textbf{kickBallToPoint()}, which provided all the functionality to go to and grab a ball, face a point and shoot; and \textbf{goToFast()}, which makes a robot go to a point as fast as possible, driving forwards or backwards to minimise delay. I also spent a lot of time developing a method for projecting ball prediction data onto a robot's facing axis, in order to efficiently find an intersection point for the robot to move when blocking balls, passing, and estimating data from opposing robots facing angles.

I also introduced our team's first non-blocking, state-based strategy design during milestone 3, which was used all the way until our final match.

%----------------------------------------------------------------------------------------
%	SECTION 6 (Conclusion)
%----------------------------------------------------------------------------------------

\section{Conclusion}

<Conclusion>

%----------------------------------------------------------------------------------------
%	SECTION 7 (Appendix)
%----------------------------------------------------------------------------------------

\section{Appendix}

<Appendix>

\end{document}



