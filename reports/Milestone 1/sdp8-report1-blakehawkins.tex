%%%%%%%%%%%%%%%%%%%%%%%%%%%%%%%%%%%%%%%%%
% Original author:
% Linux and Unix Users Group at Virginia Tech Wiki 
% (https://vtluug.org/wiki/Example_LaTeX_chem_lab_report)
%
% License:
% CC BY-NC-SA 3.0 (http://creativecommons.org/licenses/by-nc-sa/3.0/)
%
%%%%%%%%%%%%%%%%%%%%%%%%%%%%%%%%%%%%%%%%%

%----------------------------------------------------------------------------------------
%	PACKAGES AND DOCUMENT CONFIGURATIONS
%----------------------------------------------------------------------------------------

\documentclass[a4paper,12pt]{article}

\usepackage{graphicx} % Required for the inclusion of images
\usepackage{titling}

\setlength{\droptitle}{-13em}   % This is your set screw

\setlength\parindent{0pt} % Removes all indentation from paragraphs

\renewcommand{\labelenumi}{\alph{enumi}.} % Make numbering in the enumerate environment by letter rather than number (e.g. section 6)

\usepackage{times} % Uncomment to use the Times New Roman font

%----------------------------------------------------------------------------------------
%	DOCUMENT INFORMATION
%----------------------------------------------------------------------------------------

\title{SDP Group 8: Milestone 1 Individual Report} % Title

\author{Blake Hawkins} % Author name

\date{29 January, 2014} % Date of the milestone

\begin{document}

\maketitle % Insert the title, author and date

\begin{center}
Mentor: Katharina Heil % Instructor/supervisor
\\
Members: Blake Hawkins, % Partner names
Lubomir Vikev,
Borislav Ikonomov,
Yordan Stoyanov,
James Linehan,
Lukas Dirzys,
Iain Brown,
Emanuel Martinov,
Robaidh Mackinnon,
Aneesh Ghosh

\end{center}

%----------------------------------------------------------------------------------------
%	SECTION 1 (Contribution Summary)
%----------------------------------------------------------------------------------------

\section{Contribution Summary}

Initially I spent a lot of time together with Lubomir doing management related tasks. I read much of the NXJ tutorials and scanned the code archive, creating a document describing the advantages of the code in the archive and a list of packages used. From here I switched gears to building a module for odometry for milestone 1, which was flawed due to NXJ imprecision but still useful. Lastly I fixed a few bugs in the vision system.
 
%----------------------------------------------------------------------------------------
%	SECTION 2 (High Points)
%----------------------------------------------------------------------------------------

\section{High Points}

\begin{itemize}
\item Odometry System backbone
\item Management
\item Refactoring/Bug Fixes
\end{itemize}

%----------------------------------------------------------------------------------------
%	SECTION 3 (Improvement Points)
%----------------------------------------------------------------------------------------

\section{Improvement Points}

\begin{itemize}
\item Avoid spending too much time on menial tasks - recognize what is a time sink sooner
\item Improve my own activity on Trac
\item Discuss tasks with others
\end{itemize}

%----------------------------------------------------------------------------------------
%	SECTION 4 (Self Assessment, Exceptional Contributions)
%----------------------------------------------------------------------------------------

\section{Self Assessment, Exceptional Contributions}

I would award myself a 3/5, because although I spent a significant amount of time on the project, I wasted too much of that time on tasks that were not meaningful enough to my group.\\
\\
I would award a 4/5 to both Lubomir Vikev and James Linehan, because they not only did similar amounts of work to myself, but also consistently contributed meaningful work, including a vision system backbone and the entire robotics for M1 and M2.

%----------------------------------------------------------------------------------------
%	SECTION 6 (Extras, comment out to remove)
%----------------------------------------------------------------------------------------

% \section{Extra Comments}




\end{document}



